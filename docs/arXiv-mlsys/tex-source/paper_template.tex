%%%%%%%% ML-SYS EXAMPLE LATEX SUBMISSION FILE %%%%%%%%%%%%%%%%%

\documentclass{article}

% Recommended, but optional, packages for figures and better typesetting:
\usepackage{microtype}
\usepackage{graphicx}
% \usepackage{subfigure}
\usepackage{booktabs} % for professional tables

%%%%%%%%%%%%% PACKAGE
\usepackage{tabularx}
\usepackage{array}
\setlength{\tabcolsep}{3pt} % Adjust horizontal padding if needed
\usepackage{makecell}

\usepackage{tcolorbox}
\tcbuselibrary{listingsutf8}

\usepackage{url}
\def\UrlBreaks{\do\/\do-}
\usepackage{color}
\usepackage{listings} % Make sure this is included

\usepackage[frozencache,cachedir=minted-cache]{minted}

% \usepackage[noend]{algorithmic}
\usepackage{mdframed}
% \usepackage{sectsty}
% \usepackage{algorithm}
% \usepackage{algpseudocode}

\usepackage{eqparbox}
% subfigures
\usepackage{caption}
\usepackage{subcaption}
\usepackage{tikz}
\usepackage{amsmath}
\usepackage{amsthm,thmtools}
\usepackage{multirow}
\newcommand*\circled[1]{\tikz[baseline=(char.base)]{
            \node[shape=circle,draw,inner sep=1pt] (char) {#1};}}

\declaretheoremstyle[
  headfont=\bfseries,
  bodyfont=\itshape,
]{myplain}
\declaretheoremstyle[
  headfont=\bfseries,
  bodyfont=\normalfont,
]{mydefinition}
\declaretheoremstyle[
  headfont=\itshape,
  bodyfont=\normalfont,
]{myremark}

\declaretheorem[
  style=myplain,
  name=Theorem,
]{theorem}
\declaretheorem[
  style=mydefinition,
  name=Problem,
]{problem}
\declaretheorem[
  style=myremark,
  unnumbered,
  name=Solution,
]{solution}

\setminted[sql]{breaklines, framesep=3mm, fontsize=\footnotesize, numbersep=5pt}

% Define your colors
\definecolor{codegreen}{rgb}{0,0.6,0}
\definecolor{codegray}{rgb}{0.5,0.5,0.5}
\definecolor{codepurple}{rgb}{0.58,0,0.82}
\definecolor{backcolour}{rgb}{0.95,0.95,0.92}

% Define the listing style
\lstdefinestyle{mystyle}{
    backgroundcolor=\color{backcolour},
    commentstyle=\color{codegreen},
    keywordstyle=\color{magenta},
    numberstyle=\tiny\color{codegray},
    stringstyle=\color{codepurple},
    basicstyle=\ttfamily\footnotesize,
    breakatwhitespace=false,
    breaklines=true,
    captionpos=b,
    keepspaces=true,
    numbers=left,
    numbersep=5pt,
    showspaces=false,
    showstringspaces=false,
    showtabs=false,
    tabsize=2,
    lineskip=-1pt
}


% hyperref makes hyperlinks in the resulting PDF.
% If your build breaks (sometimes temporarily if a hyperlink spans a page)
% please comment out the following usepackage line and replace
% \usepackage{mlsys2024} with \usepackage[nohyperref]{mlsys2024} above
\usepackage{hyperref}

% Attempt to make hyperref and algorithmic work together better:
\newcommand{\theHalgorithm}{\arabic{algorithm}}

% Use the following line for the initial blind version submitted for review:
% \usepackage{mlsys2025}
% \pagestyle{empty}


% If accepted, instead use the following line for the camera-ready submission:
\usepackage[accepted]{mlsys2025}
\fancyfoot[C]{\raisebox{-20pt}{\thepage}} % to show page number at bottom center without affecting the text layout

% The \mlsystitle you define below is probably too long as a header.
% Therefore, a short form for the running title is supplied here:
\mlsystitlerunning{Improving GGR Algorithm}

\begin{document}
\lstset{style=mystyle}

\twocolumn[
\mlsystitle{Improving Optimality and Speed of Greedy Group Recursion Algorithm}

% It is OKAY to include author information, even for blind
% submissions: the style file will automatically remove it for you
% unless you've provided the [accepted] option to the mlsys2024
% package.

% List of affiliations: The first argument should be a (short)
% identifier you will use later to specify author affiliations
% Academic affiliations should list Department, University, City, Region, Country
% Industry affiliations should list Company, City, Region, Country

% You can specify symbols, otherwise they are numbered in order.
% Ideally, you should not use this facility. Affiliations will be numbered
% in order of appearance and this is the preferred way.
\mlsyssetsymbol{equal}{*}

%\begin{mlsysauthorlist}
%\mlsysauthor{Shu Liu}{equal,to}
%\mlsysauthor{Asim Biswal}{equal,to}
%\mlsysauthor{Luis Gaspar Schroeder}{to,goo}
%\mlsysauthor{Ion Stoica}{to}
%\mlsysauthor{Matei Zaharia}{to}
%\end{mlsysauthorlist}
\begin{mlsysauthorlist}
\mlsysauthor{Florin Dobrian}{dag}
\mlsysauthor{Oleg Puzyrko White}{dag}
\end{mlsysauthorlist}

%\mlsysaffiliation{to}{UC Berkeley}
%\mlsysaffiliation{goo}{Technical University of Munich}
%\mlsysaffiliation{ed}{Stanford University}
\mlsysaffiliation{dag}{Data Analytics Group}

\mlsyscorrespondingauthor{Oleg P. White}{oleg.p.white@gmail.com}
% \mlsyscorrespondingauthor{Eee Pppp}{ep@eden.co.uk}

% You may provide any keywords that you
% find helpful for describing your paper; these are used to populate
% the "keywords" metadata in the PDF but will not be shown in the document
\mlsyskeywords{Machine Learning, MLSys}

\vskip 0.3in

% V2
\begin{abstract}

Recently~\cite{liu2025optimizingllmqueriesrelational} has presented efficient
algorithm - Greedy Group Recursion (GGR) - for reordering the rows and the fields
within each row of an input table to maximize key-value (KV) cache reuse when
performing LLM serving.
In this paper, we propose several adjustments to GGR algorithm that can improve
optimality of the solution and reduce its execution time.

\end{abstract}

]



% this must go after the closing bracket ] following \twocolumn[ ...

% This command actually creates the footnote in the first column
% listing the affiliations and the copyright notice.
% The command takes one argument, which is text to display at the start of the footnote.
% The \mlsysEqualContribution command is standard text for equal contribution.
% Remove it (just {}) if you do not need this facility.

\printAffiliationsAndNotice{}  % leave blank if no need to mention equal contribution
%\printAffiliationsAndNotice{\mlsysEqualContribution} % otherwise use the standard text.


\section{Introduction}

There has been growing research on LLM inference optimization.
In particular, recent
work~\cite{liu2025optimizingllmqueriesrelational, cheng2025letbarbariansinai}
presents solutions to optimize relational data analytics workloads for offline LLM inference.
It proposes Greedy Group Recursion (GGR), an approximate
algorithm that leverages functional dependencies (such as
primary and foreign key relationships from the data schema)
and table statistics, which are readily available in many
databases and analytics systems, to reduce the search space.


\bibliography{reference}
\bibliographystyle{mlsys2025}

%% If your work has an appendix, this is the place to put it.
% \appendix
% \input{sections/fixed_proof}

\end{document}
