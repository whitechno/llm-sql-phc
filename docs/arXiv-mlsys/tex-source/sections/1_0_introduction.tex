\section{Introduction}

There has been growing research on LLM inference optimization. In particular, recent
work~\cite{liu2025optimizingllmqueriesrelational, cheng2025letbarbariansinai}
presents solutions to optimize relational data analytics workloads for offline LLM inference.
It proposes Greedy Group Recursion (GGR), an approximate algorithm that leverages
functional dependencies (such as primary and foreign key relationships from the data
schema) and table statistics, which are readily available in many databases and
analytics systems, to reduce the search space.

% @formatter:off
\begin{algorithm}[t!]
\caption{Greedy Group Recursion (GGR)}
\begin{algorithmic}[1]
\small
\STATE \textbf{Input:} Table $T$, Functional Dependency $FD$
\STATE \textbf{Output:} Prefix Hit Count $S$, Reordered List of Tuples $L$


\item[]
\FUNCTION{$\textsc{HitCount} (v, c, T, FD)$}
    \STATE $R_v \gets \{i \mid T[i,c] = v\}$
    \STATE $\text{inferred\_cols} \gets \{c' \mid (c, c') \in FD\}$
    \STATE $\text{inferred\_vals} \gets \{T[R_v[1], c'] \mid c' \in \text{inferred\_cols}\}$
    \STATE $\text{tot\_len} \gets \text{len}(v)^2 + \sum_{\substack{c' \in \text{inferred\_cols}}} \left( \frac{\sum_{r \in R_v} \text{len}(T[r, c'])}{|R_v|} \right)^2$
    \STATE $HC \gets \text{tot\_len} \times (|R_v| - 1)$
    \STATE $cols \gets [c] + \text{inferred\_cols}$
    \STATE $vals \gets [v] + \text{inferred\_vals}$
    \STATE \textbf{return} $HC, cols, vals$
\ENDFUNCTION


\item[]
\FUNCTION{\textsc{GGR}($T$, $FD$)}
    % \IF{$|T|_{rows} = 1$ or $|T|_{cols} = 1$}
    %     \STATE \textbf{return } \text{Base case processing as in \optimal}
    % \ENDIF
    \IF{$|T|_{rows} = 1$}
        \STATE {\bfseries return} $0, [T[1]]$
    \ENDIF
    \IF{$|T|_{cols} = 1$}
        \STATE $S \gets \sum_{v \in \text{distinct}(T[,1])} \textsc{HitCount}(v, 1, T)$ % groupby or sort, choose best group, append
        \STATE {\bfseries return} $S, sort([T[i] \mid i \in 1 \dots |T|_{rows}])$
    \ENDIF

    \STATE $b\_v, b\_c \gets \text{None}, \text{None}$
    \STATE $max\_HC, b\_cols, b\_vals \gets -1, [], []$

    \FOR{$c \in \text{columns}(T)$, $v \in \text{distinct}(T[,c])$}
        \STATE $HC, cols, vals \gets \textsc{HitCount}(v, c, T, FD)$
        \IF{$HC > max\_HC$}
            \STATE $b\_v, b\_c \gets v, c$
            \STATE $max\_HC, b\_cols, b\_vals \gets HC, cols, vals$
        \ENDIF
    \ENDFOR

    \STATE $R\_v \gets \{i \mid T[i, b\_c] = b\_v\}$
    \STATE $A\_HC, A\_L \gets \textsc{GGR}(T[\text{rows} \setminus R\_v, \text{cols}], FD)$
    \STATE $B\_HC, B\_L \gets \textsc{GGR}(T[R\_v, \text{cols} \setminus b\_cols], FD)$
    \STATE $C\_HC \gets max\_HC$
    \STATE $S \gets A\_HC + B\_HC + C\_HC$
    \STATE $L \gets [b\_vals + B\_L[i] \mid i \in 1 \dots |R\_v|] + A\_L$
    \STATE \textbf{return} $S, L$
\ENDFUNCTION

\item[]
\STATE \textbf{return} \textsc{GGR}($T$, $FD$)
\end{algorithmic}
\label{alg:greedy}
\end{algorithm}
% @formatter:on


\vspace{-0.5em}

\subsection{Greedy Group Recursion (GGR) Algorithm} \label{sec:ggr}

The GGR algorithm is described in detail in~\cite{liu2025optimizingllmqueriesrelational}.
Let us briefly outline its main points.

The pseudocode specification of GGR is in Algorithm~\ref{alg:greedy} listing and is
taken from the original paper but with several critical corrections
(and multiple little typo fixes).

GGR algorithm optimizes LLM queries by finding a ($n \times m$) table rearrangement
that maximizes the LLM's KV cache prefix hit count (PHC).
It achieves this goal by reordering the rows and the fields within each row to
\textbf{maximize} the \textit{prefix hit count}.
Each row may have a different field order.
The rearranged table is represented as a list of tuples $L$, where each tuple
in $L$ corresponds to a row, and the tuple elements contain the column values.
A cell in the list of tuples is denoted as $L[r][f]$, indicating the value in tuple $r$ at position $f$.

The hit count of a single cell $L[r][c]$ is non-zero \textbf{only} if its value is
the same as in the previous row $L[r][c] = L[r-1][c]$ \textbf{and} all preceding
fields have the same property $\forall f \leq c, L[r][f] = L[r-1][f]$.
Then its hit count is computed as the square of the value's string length:

\vspace{-1.5em}
\begin{equation}
    \textit{hit}(L, r, c) =
    \begin{cases}
        \text{len}(L[r][c])^2 & \text{if } \forall f \leq c, \\ & L[r][f]= \\ & L[r-1][f] \\
        0 & \text{otherwise}
    \end{cases}
    \label{eq:cellhit}
\end{equation}
\vspace{-1.5em}

The squared string length reflects the quadratic complexity of token processing in
LLM inference, where each token computation depends on every preceding token and
increases computational cost quadratically with the input length.

The hit count for a single row $r$ in $L$ is then just a sum of hit counts for all
cells in a row:
\vspace{-1.5em}

\begin{equation}
    \textit{hit}(L, r) = \sum_{c=1}^{m} \textit{hit}(L, r, c)
    \label{eq:rowhit}
\end{equation}
\vspace{-1.5em}

The PHC for a list of tuples $L$ with $n$ rows and $m$ fields is given by:
\vspace{-1.5em}

\begin{equation}
    \text{PHC}(L) = \sum_{r=1}^{n} \textit{hit}(L, r)
    \label{eq:phc}
\end{equation}
\vspace{-1.5em}

The $\text{PHC}(L)$ is the objective function of a table reordering algorithm: it
tries to find output $L$ with maximum $\text{PHC}(L)$ value.

The greedy part of GGR is based on finding a distinct value $b\_v$
(in the corresponding column $b\_c$) with the highest hit count and then
rearranging the table so that all cells with that value
appear in consecutive rows and in the first field of the output $L$.
GGR then uses the group of $b\_v, b\_c$ cells to split the table into two smaller
sub-tables and recursively run GGR on each of them.
See Figure~\ref{fig:optimal} for illustration of this key step.

Compared to the brute-force algorithm that
requires $n!*(m!)^n$ potential orderings, GGR significantly reduces the search
space by selecting the highest-hit value and then reducing the dimensions of the
table at each recursive step with the maximum depth of recursion $O(\min(n,m))$.
GGR achieves close-to-perfect PHC output with fast practical execution time.

\vspace{-0.5em}

\subsubsection{Functional Dependencies}

GGR algorithm leverages a table's functional dependencies (FD) to reduce the number
of fields it needs to consider at each recursion step.

We say that two columns $A$ and $B$ are in functional dependency $A \leftrightarrow B$
in a table $T$ if, for any two rows $r_1$ and $r_2$, $T[r_1][A] = T[r_2][A]$ implies
$T[r_1][B] = T[r_2][B]$ and vice versa.

In other words, for any distinct value $a$ in the column $A$, there is a distinct
value $b$ in the column $B$ such that they both are on the same rows.
I.e., for $R_a \gets \{i \mid T[i,A] = a\}$ and $R_b \gets \{i \mid T[i,B] = b\}$ we have
$R_a = R_b$.

FD rules can be specified as a list of disjoint sets containing column indices.
For example, for the table with columns $A, B, C, D, E, F$ where
$A \leftrightarrow B$ and $C \leftrightarrow D \leftrightarrow E$, the FD rules can be
specified as $[[A, B], [C, D, E]]$.

\vspace{-0.5em}

\subsubsection{GGR Specification}

We can follow the pseudocode specification of GGR in Algorithm~\ref{alg:greedy} listing.

At each recursive step, the GGR algorithm scans the table (lines 21–29) to find
all distinct values with corresponding hit counts (lines 3–12).
It then selects the highest-hit value $b\_v$ (in the column $b\_c$) and splits the
table into two sub-tables - one with all the rows $R\_v$ containing $b\_v$ value
but excluding the column $b\_c$ and its FD-associated columns
(line 31: $T[R\_v, \text{cols} \setminus b\_cols]$),
and another sub-table with the remaining rows
(line 32: $T[\text{rows} \setminus R\_v, \text{cols}]$).
In the final step, GGR recurses on the two sub-tables (lines 31–32) and calculates
the total PHC as the sum of PHCs computed for each sub-table and of hit count
computed for $b\_v, b\_c$ (line 34).

\begin{figure}
    \centering
    \includegraphics[width=0.49\textwidth]{figures/MLSys_Figures/prefix_hit_maximization}
    \vspace{-1.5em}
    \caption{GGR picks the group with the maximum hit count at each step and calculates PHC
    as the sum of PHC of the elected group values (yellow box),
        the sub-table $T$ excluding rows $R_v$ (green box),
        and the sub-table of rows $R_v$ excluding the field where the value is located in (blue box).}
    \label{fig:optimal}
    \vspace{-1.5em}
\end{figure}
